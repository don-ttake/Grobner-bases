\documentclass[11pt, oneside]{article}   	
\usepackage[margin=1in]{geometry}                		
\geometry{letterpaper}                   		
\usepackage{graphicx}						
\usepackage{amssymb}
\usepackage{amsmath}
\usepackage{amsthm}
\usepackage{enumerate}

%Commands above this line set up the type of document, and ensure it has access to the LaTeX files needed to understand your commands.

%Here, I define some "shortcuts" for notation I commonly use.
\newcommand{\N}{\mathbb N}
\newcommand{\Z}{\mathbb Z}
\newcommand{\Q}{\mathbb Q}
\newcommand{\C}{\mathbb C}
\newcommand{\R}{\mathbb R}
\newcommand{\F}{\mathbb F}

\newcommand{\stab}{\operatorname{stab}}
\newcommand{\orb}{\operatorname{orb}}
\newcommand{\Aut}{\operatorname{Aut}}
\newcommand{\Inn}{\operatorname{Inn}}
%\newcommand{\char}{\operatorname{char}}

\newtheorem*{proposition}{Proposition}
\newtheorem*{theorem}{Theorem}

\title{Homework 8}
\author{The Author}
%\date{}			




%THIS IS WHERE THE ACTUAL TEXT OF THE DOCUMENT BEGINS.				
\begin{document}

\begin{center}\noindent{\bf Math 333:  Homework \#8}\\Mingchen Li\\ \end{center}
\thispagestyle{empty}



\hrulefill %This just makes a nice horizontal line. Useful if you like to separate your problems with lines!





\begin{enumerate}

\item[{\bf 13.35}] Let $F$ be a field of order $2^n$. Prove that $\operatorname{char} F = 2$.
\begin{proof}
Since by definition a field is a commutative ring with unity in which every non-zero element is a unit. Thus $F$ has unity 1. And by Theorem 13.3 on the Gallian, we know the characteristic of a ring is the additive order of the unity. Thus we aim to show $|1|=2$

By Lagrange's Theorem, we know that the additive order of 1 must divide $2^n$. Since finite fields are integral domain, and integral domain have characteristic of either 0 or prime, $|1|$ is a prime that divides $2^n$ which leaves us to the only option of 2.
\end{proof}

\newpage
\item[{\bf 13.51}] Prove that every finite field has order $p^n$, where $p$ is prime. (Hint: use facts about finite Abelian groups. Be careful though...these are facts about {\it groups} only.)
\begin{proof}
Let $F$ be an arbitrary finite field. We know finite field is integral domain. Thus by Theorem 13.4 in Gallian, we know that a characteristic of $|1|$ is prime or 0. Since $F$ is a finite field, the additive order of $1$ will not be infinite, thus $|1|=p$ for some p. 

Then we aim to show that any other non-zero element of $F$ has additive order of p as well. For arbitrary $a,b\in F$, let $|a|=n\neq 0$, $|b|=m$. Since the field is distributive respect to addition:
\[(n\cdot a)b=ab+ab+ab+...ab=a(n\cdot b)=0\]
Assume that $n\neq m$, then given that $a,b$ are not zero divisors, $n\cdot b=0$. However this contradicts with the assumption as $n\neq m$ and the order is unique. Thus $a,b$ must have the same additive order.

Thus for every other nonzero element of $F$, it must share the same additive order with 1. Since $F$ is also a commutative group under addition. The order of $F$ must divide the order of element , thus the order of $F $ can only be multiple of $p$. Thus $|F|=p^n$ for some positive n. 
\end{proof}

\newpage
\item[{\bf 13.46}] Suppose that $a$ and $b$ belong to a commutative ring $R$ and $ab$ is a zero divisor. Show that either $a$ or $b$ is also a zero divisor.
\begin{proof}
Given that $ab$ is a zero divisor, we know that there is some $c\in R$ such that $abc=0_R$. Since multiplication is associative, $a(bc)=0_R$ as well, where $c\neq 0$. If $bc=0$, then $b$ is a zero divisor. If $bc\neq 0$, then $a$ is a zero divisor. Thus either $a$ or $b$ is a zero divisor. 
\end{proof}



\newpage
\item[{\bf Problem 4:}] In this problem, you will consider the idea of an ideal {\it generated by} a finite set. Throughout, let $R$ be a commutative ring with unity.

\begin{enumerate}
\item Let $a\in R$. Prove that the set $$\langle a \rangle = \{ra\,|\, r\in R\}$$ is an ideal of $R$, called the {\it principal ideal} of $R$ generated by $a$.  
\begin{proof}
Take arbitrary $r_1a, r_2a\in \langle a \rangle$. By the distributive property of Ring, we have $r_1a-r_2a=(r_1-r_2)a\in \langle a \rangle$. 

Let $r\in R$ be arbitrary. Given that the Ring $R$ is commutative: $r_1ar=rr_1a$. At the same time, since rings are closed under multiplication, $rr_1\in R$ as well. Thus $r_1ar=(rr_1)a\in \langle a \rangle$.

Thus we have satisfied the ideal test and $\langle a \rangle$ is an ideal of $R$.
\end{proof}
\item Let $a_1,a_2,...,a_n\in R$. Prove that the set $$\langle a_1,...,a_n \rangle = \{r_1a_1+\cdots + r_na_n\,|\, r_i\in R\}$$ is an ideal of $R$, called the ideal of $R$ generated by $a_1,...,a_n$.  Prove also that if any ideal $I$ of $R$ contains all of $a_1,...,a_n$, then $I$ contains $\langle a_1,...,a_n\rangle$ (that is, prove that the ideal generated by $a_1,...,a_n$ is the {\it smallest} ideal of $R$ which contains these elements).

\begin{proof}
Let $I$ be an arbitrary ideal of $R$ that contains all of $a_1,...a_n$, then by the multiplicative property of ideal, we know that for arbitrary $r_1\in R$, $ar_1\in I$. 

In similar vein, for arbitrary $r_2...r_n\in R$, we have $r_2a_2, r_3a_3...r_na_n\in I$ as well. Since the addition is also closed in $I$, we have $r_1a_1+\cdots + r_na_n\in I$ for arbitrarily chosen $r_i\in R$. To write in set notation: 
\[\{r_1a_1+\cdots + r_na_n\,|\, r_i\in R\}\subseteq I\]
And we have found our desired results.
\end{proof}
\end{enumerate}


\newpage

\item[{\bf Problem 5}] This problem has two parts and examines in detail the ideals of $\Z$.
\begin{enumerate}
\item[{\bf 14.41:}]  An integral domain $D$ is called a {\it principal ideal domain} if every ideal of $D$ is of the form $\langle a \rangle$ for some $a\in R$. Prove that $\Z$ is a principal ideal domain. 
\begin{proof}
Let $R$ be an ideal of $\Z$, by definition of Ideal, it is also a subring of $\Z$. Notice that a ring is a commutative group under addition. Thus $R$ essentially is a subgroup of $\Z$. Notice that $\Z$ is cyclic group and by the FT of cyclic groups, every subgroup of cyclic group is also cyclic. Thus every ideal of $\Z$ is a cyclic group. 
\end{proof}

\item[{\bf 14.9}]  If $n$ is an integer greater than $1$, show that $\langle n \rangle  = n\Z$ is a prime ideal of $\Z$ if and only if $n$ is prime. (This, in part, is the motivation for the name ``prime" for such an ideal in a general ring.)
\begin{proof}
$\Rightarrow$ Assume that $\langle n \rangle  = n\Z$ is a prime ideal, then assume that $n=kp$ where $k,p$ are product of primes that are greater than 1. Then consider $k,p\in \Z$ and $kp\in \langle n \rangle$. However, neither $k$ nor $p\in \langle n \rangle$ as they are strictly less than $n$ and greater than $1$. Thus it violates the definition of prime ideal.

$\Leftarrow$ Assume that $n$ is a prime, then for arbitrary $a,b\in \Z$, $ab\in \langle n \rangle$. This implies that $ab=cn$ and by arithmetic we know either $a$ or $b$ is a multiple of $n$. Thus $a\in \langle n \rangle$ or $b\in \langle n \rangle$. $\langle n \rangle$ is hereby a prime ideal.
\end{proof}
\end{enumerate}

 \newpage
\item[{\bf 14.8}] Prove that the intersection of any set of ideals of $R$ forms an ideal of $R$.
\begin{proof}
Consider an arbitrary collection of ideals: $\mathcal{F}$. If $a\in \bigcap_{I\in  \mathcal{F}} I$, then for every $r\in R$, we have $ar,ra\in I$ for each $I\in \mathcal{F} $. Thus $ar, ra\in   \bigcap_{I\in  \mathcal{F}} I$ as well. 

Then we aim to show that $ \bigcap_{I\in  \mathcal{F}} I$ is a subring of $R$. Since each $I$ is ideal, hereby subring, we know that for arbitrary $a,b\in  \bigcap_{I\in  \mathcal{F}} I$, $a,b\in I$ as well for each $I\in \mathcal{F}$. Then for each $I$, they must pass the subring test: $a-b\in I$ and $ab\in I$ for each $I\in \mathcal{F}$. Thus $a-b, ab\in  \bigcap_{I\in  \mathcal{F}} I$. $ \bigcap_{I\in  \mathcal{F}} I$ is hereby a subring. 

Thus we have proven that the intersection of arbitrary number of ideal is an ideal as well.
\end{proof}


\newpage
\item[{\bf 14.28}] Prove that $\R[x]/\langle x^2+1\rangle$ is a field.
\begin{proof}
We aim to show that the ideal $\langle x^2+1 \rangle$ is maximal ideal in $\R[x]$. Assume that there is some ideal $A$ such that 
\[\langle x^2+1 \rangle\subseteq A \subseteq \R[x]\]
Let $f(x)\in A$ and $f(x)$ is not in $\langle x^2+1 \rangle$. Then we can express $f(x)$ as :
\[k(x)(x^2+1)+r(x)\]
Where $r(x)$ is a non-trivial remainder polynomial with degree less than 2. Thus we can write $r(x)=ax+b$ where $a,b$ are not both zero. Then we have:
\[ax+b=f(x)-k(x)(x^2+1)\]
Since $f(x)\in A$ and $k(x)(x^2+1)\in  \langle x^2+1 \rangle$ hereby in $A$ as well. Thus we have $ax+b\in A$. 

Since $A$ is ideal, meaning it is an subring, it is closed under addition and multiplication and exist additive inverse: $ax-b\in A$. Thus $a^2x^2-b^2=(ax+b)(ax-b)\in A$. By definition of $\langle x^2+1 \rangle$, we also know that $a^2(x^2+1)\in \langle x^2+1 \rangle \subseteq A$.

Thus we have find a constant non-zero polynomial in $A$, namely:
\[a^2+b^2=a^2(x^2+1)-(a^2x^2-b^2)\]
Thus we have also the constant $1$ polynomial since $A$ is ideal and $p(x)=h(x)*(a^2+b^2)=1\in A$ for $h(x)=\dfrac{1}{a^2+b^2}\in \R[x]$. 

Then for arbitrary $r\in \R[x]$, we know that $r=rp(x)\in A$ using the ideal property. Thus $A=\R[x]$ and we have shown that $\langle x^2+1\rangle$ is maximal ideal for $\R[x]$.

From this, we can use theorem 14.4 on Gallian:
\begin{theorem}
Let R be a commutative ring with unity and let A be an ideal of R, then R/A is a field if and only if A is maximal.
\end{theorem}

We can hereby conclude that $\R[x]/\langle x^2+1\rangle$ is a field.

\end{proof}

\newpage
\item[{\bf 14.57}]  If $R$ is a principal ideal domain and $I$ is an ideal of $R$, prove that every ideal of $R/I$ is principal.
\begin{proof}
For arbitrary ideal of $R/I$, it is an ideal for the set $\{r+I|r\in R\}$ with addition and multiplication defined on $r$s. Thus the ideal for $R/I$ is essentially in the form of $A/I$ where $A$ is an ideal of $R$.

By the definition of principal ideal domain: every ideal of $R$ has the form of $\langle a \rangle=\{ad|d\in R\}$. Thus $A/I=\langle a \rangle/I=\{ra/I|r\in R\}=(a/I)(R/I)$. $A/I$ is hereby generater by single element of $a/I$ and hereby principal.
\end{proof}
\end{enumerate}

\end{document}  

%THIS IS WHERE THE DOCUMENT ENDS. Anything written after this will not appear on the pdf.