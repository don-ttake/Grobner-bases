\documentclass[11pt, oneside]{article}   	
\usepackage[margin=1in]{geometry}                		
\geometry{letterpaper}                   		
\usepackage{graphicx}						
\usepackage{amssymb}
\usepackage{amsmath}
\usepackage{amsthm}
\usepackage{enumerate}

%Commands above this line set up the type of document, and ensure it has access to the LaTeX files needed to understand your commands.

%Here, I define some "shortcuts" for notation I commonly use.
\newcommand{\N}{\mathbb N}
\newcommand{\Z}{\mathbb Z}
\newcommand{\Q}{\mathbb Q}
\newcommand{\C}{\mathbb C}
\newcommand{\R}{\mathbb R}
\newcommand{\F}{\mathbb F}

\newtheorem*{proposition}{Proposition}
\newtheorem*{theorem}{Theorem}

\title{Homework 3}
\author{The Author}
%\date{}			




%THIS IS WHERE THE ACTUAL TEXT OF THE DOCUMENT BEGINS.				
\begin{document}

\begin{center}\noindent{\bf Math 333:  Homework \#3}\\Mingchen Li\\ \end{center}
\thispagestyle{empty}
\hrulefill %This just makes a nice horizontal line. Useful if you like to separate your problems with lines!



\begin{enumerate}

\item[{\bf DF2.4.13:}] We have extensively considered cyclic subgroups - those generated by a single element $\langle a \rangle$. In this problem (and the next) we will look at subgroups generated by {\it any} set of elements.

{\bf Definition:} Let $G$ be a group, and $A\subseteq G$, $A\neq \emptyset$. The subgroup of $G$ {\it generated by} $A$, denoted $\langle A \rangle$, is the intersection of all subgroups $H\leq G$ containing $A$. That is, $$\langle A \rangle = \bigcap_{A\subseteq H,  H\leq G} H.$$

(On the previous homework, you explained why an arbitrary intersection of subgroups would form a subgroup, so there is no need to prove this is a subgroup.) Another way to think of this is that $\langle A \rangle$ is the smallest possible subgroup of $G$ which contains all of $A$.

\begin{itemize}
\item Show that if $H$ is a subgroup of $G$, then $H=\langle H \rangle$. 

\begin{proof}
It is fairly obvious for $H\subseteq \langle H \rangle$ since by definition $\langle H \rangle$ is the smallest subgroup that contain H. 

By definition, $\langle H \rangle= \bigcap_{H\subseteq K,  K\leq G} K$. Since H is a subgroup, it is one of the K's. Thus take arbitrary $a\in \langle H \rangle: a\in H$ as well. Thus $H\supseteq \langle H \rangle$. 
\end{proof}

\item Prove that the group of positive rational numbers under multiplication is generated by the set $$A=\left\{\frac{1}{p}\,\bigg|\, p \text{ is a prime }\right\}.$$
\begin{proof}
We shall first denote the group of positive rational number under multiplication in a more formal way:
\[B=\left\{\dfrac{q}{p}\bigg|\, \text{gcd(p,q)=1}, p,q\in \N \right\}\]
We aim to show that $(B, *)$ is the smallest group that contain A. First we need to show $A\subseteq B$. Take arbitrary $a\in A$, $a=\dfrac{1}{p}$ where p is a prime. Thus $gcd(1, p)=1, p\in \N, p\neq 0. $ This proves that $a\in B$.

Then we aim to show that $(B, *)$ is the smallest group. We shall first find the elements that are in $\langle A \rangle$: \begin{enumerate}
    \item Since elements of A are in a group, then any multiple of its elements are in the group as well, namely:
        \[\left\{ \dfrac{1}{\hat{p}} \bigg|\, \hat{p}\text{ is product of primes} \right\}\]
    \item In similar vein, the elements of A must have their multiple inverses: $(\dfrac{1}{p})^{-1}=p$ that are also in the group. By (a), the multiple of inverses are also in the group, namely:
    \[\left\{ \hat{q} \bigg|\, \hat{q}\text{ is product of primes} \right\}\]
    
    Since the group is closed under multiplication, also by the Fundamental Theorem of Arithmetic: "every positive integer greater than one can be written uniquely as a product of primes", we can combine both sets into one:
    \[\left\{\dfrac{q}{p}\bigg|\, \text{gcd(p,q)=1}, p,q\in \N \right\}\]
    Which turns out to be set B. Thus B is the smallest possible group that contain A. 

    
\end{enumerate}


    
\end{proof}
\end{itemize}

\newpage
\item [{\bf DF 2.4.14:}] A group $H$ is called {\it finitely generated} if there is a finite set $A$ such that $H=\langle A\rangle$. 
\begin{enumerate}
\item  Prove that every finite group is finitely generated. 
\begin{proof}
By definition: $\langle A \rangle = \bigcap_{A\subseteq \hat{H},  \hat{H}\leq G} \hat{H}=H$. Since H is finite and $A\subseteq H$, thus A must have less or equal elements than H, A is finite. 
\end{proof}
\item Prove that $\Z$ is finitely generated.
\begin{proof}
Consider $A=\{1\}$. We aim to show $\Z= \langle A \rangle$. We know that $\Z$ is a cyclic group generated by 1. Thus $\Z= \langle A \rangle$
\end{proof}
\item Prove that every finitely generated subgroup of the additive group $\Q$ is cyclic.\footnote{If $H$ is a finitely generated subgroup of $\Q$, show that $H\leq \langle \frac{1}{k}\rangle$, where $k$ is the product of all denominators which appear in a set of generators for $H$.}

\begin{proof}
Let H be an arbitrary finitely generated group of $\Q$: $H=\langle A \rangle$. We aim to show $H\leq \langle \frac{1}{k}\rangle$, where $k$ is the product of all denominators which appear in a set of generators for $H$. Under the same operation, $\langle \frac{1}{k}\rangle$ form a cyclic group. For arbitrary $h\in H$, h can be expressed as a linear combination of elements of A. Since each element of A is a multiple of $\frac{1}{k}$: $h=\frac{p}{k}$ where p is a multiple of k. Thus $h\in \langle \frac{1}{k}\rangle$. $H\leq \langle \frac{1}{k}\rangle$.

Since the subgroup of cyclic group is also cyclic, H is cyclic. 
\end{proof}
\item Prove that $\Q$ is not finitely generated.
\begin{proof}
Assume $\Q$ is finitely generated, $\Q=\langle A \rangle$. Using the same reasoning and setting above, $\langle A \rangle \leq \langle \frac{1}{k}\rangle$. This shows $\Q$ is cyclic under addition which is not true. $\Rightarrow\Leftarrow$.
\end{proof}
\end{enumerate}

\newpage
\item[{\bf 4.76}] (From the Math GRE 2008 Practice Exam) If $x$ is an element of a cyclic group of order $15$ and exactly two of $x^3, x^5$, and $x^9$ are equal, determine $|x^{13}|$.

\begin{proof}
Since in a finite cyclic group, the order of each element divides order of the group, the order of x is limited to $\{1, 3, 5,15\}$. Thus $x^3\neq x^5$, $x^5 \neq x^9$. This leave the order of $x$ to be 3 and $x^3=(x^3)^3=x^9=e$. $|x^{13}|= |(x^3)^4x|=|x|=3$.
\end{proof}

\newpage
\item[{\bf 4.82:}] Let $G$ be the set of all polynomials of the form $ax^2+bx+c$ with $a,b,c\in \Z/3\Z$. We can make $G$ be a group under addition by adding the polynomials in the usual way, except that we add the coefficients modulo 3. Under this operation, prove that $G$ is a group of order 27 that is {\it not} cyclic.

\begin{proof}
Take arbitrary $g\in G$, there are 3 possible constant before each power. The number of possibility is equivalent to drawing 1 out of 3 element with replacement. Thus there are $3^3=27$ possibilities. $|G|=27$. 

Assume that $G= \langle g \rangle$ where $g\in G$, then for arbitrary $f\in G$, $f=ax^2+bx +c = g^n$. Then consider the case of $k=ax^2+bx +c+1$. $k\in G$ for obvious reason. Thus $k=g^m$. However since "a" did not change, the difference between m and n must be a multiple of $|a|$. Since $\Z/3Z$ is cyclic, $|a| \in \{1, 3\}$. At the same time $a\neq1$ since if $a=e$, $g$ cannot map all of $G$. Thus $|a|=3$. However, using the same reasoning, $|c|=3$ as well. $g^m= g^n +g^{3x}= ax^2+ \hat{b}x +c \neq ax^2+bx +c+1$. $\Rightarrow\Leftarrow $
\end{proof}
\newpage
\item[{\bf 5.23}] Show that if $H$ is a subgroup of $S_n$, then either every member of $H$ is an even permutation or exactly half of the members are even.
\begin{proof}
Since $H\leq S_n$, H has an inverse. Take arbitrary element $h\in H$, if $h$ is even(or odd), then there is $h^{-1}$ such that $hh^{-1}=e=even$. Thus $h^{-1}$ must be the same "type" as $h$. It is impossible for all element to be odd since identity is even. So we are left with two possibilities:\begin{enumerate}
    \item All elements are even. This is possible when every element is even. Such as $H=\{e\}$
    \item There are both odd and even number of elements: Here we aim to show that the number of odd and even element must be equal: take $x\in H$ to be an odd element, then for $\forall y \in H$ that are odd as well, including x itself: $xy$ is even. Thus $\{\text{odd element}\}\subseteq\{\text{even element}\}$. 
    
    In similar vein, take a odd element $\hat{x}\in H$. $\forall \hat{y}\in H$ that are even: $\hat{x}\hat{y}$ is odd and each $\hat{x}\hat{y}$ is unique. Thus  $\{\text{odd element}\}\supseteq\{\text{even element}\}$. 
\end{enumerate}
\end{proof}
\newpage
\item[{\bf 5.41}:] Suppose $\beta$ is a 10-cycle in a permutation group $S_n$. For which integers $i$ between $2$ and $10$ is $\beta^i$ also a 10-cycle?
\begin{proof}
Take i to be arbitrary in $\N[2,10]$, notice that when multiplying the same cycle $\beta$ for i times, it is essentially pushing the $k-th$ element in the cycle to $(k+i) \text{ mod } 10 -th$ location. To achieve another 10 cycle, we are essentially finding an $i$ that can map all 10 elements after performing modular addition on each element in the original cycle with $i$. In other words, any generator of $\Z/10\Z$ in $[2,10]$ will do. 

By previous knowledge, we know $\Z/10\Z$ = $\langle 1\rangle$ and $|1|=10$. By Theorem 4.2 in book we aim to find k such that:
\[\langle 1^k \rangle= \langle 1^{\text{gcd}(10,k)} \rangle = \langle 1^1\rangle\]
ie. k that are relative prime to 10 which are $\{3, 7, 9\}$.

\end{proof}
\newpage
\item[{\bf DF 1.3.18:}] Find all numbers $n$ such that $S_5$ contains an element of order $n$. 
\begin{proof}
By Theorem 5.3, 
\begin{theorem}
  The order of a permutation of a finite set written in disjoint cycle form is the least common multiple of the lengths of the cycles.
\end{theorem}
For element in $S_5$, if it has only one cycle, then the length of that cycle will be its order. Since permutation is bijection. Each cycle has length at most 5. Thus the order can be $\N[1,5]$.

If the element is compose with multiple cycles, when written in disjoint form, the sum of their length will be at most five. This yields very limited possibilities and we are only going to point out the ones that have order more than 5, namely a element composed by a cycle of length 2 and a cycle of length 3. This type of element has order of 6. 

Thus n can be any number in $\N[1,6]$

\end{proof}
\newpage
\item[{\bf 5.66:}] Show that for $n\geq 3$, the center of $S_n$ is $\{e\}$. (That is, the only element which commutes with EVERY other element is the identity.)

\begin{proof}
Assume that there is some other element, $a\neq e$, that is the center of $S_n, n\geq 3$. We can express $a$ as a product of disjoint cycles $k$. Then take one of the cycles that is at least longer than 1. This cycle exist because if all cycles are identity cycle, $a=e$ which contradict our assumption. We can express $k$ as:
\[k=(k_1 k_2 ...k_n) \]
Then consider the permutation $t=(k_3k_2)$ where $k_3 \neq k_1\neq k_2$. Such $k_3$ exist since $n\geq3$. Then $kt(k_1)=k_2\neq tk(k_1)=k_3. \Rightarrow\Leftarrow$ 
\end{proof}
 

\end{enumerate}

\end{document}  

%THIS IS WHERE THE DOCUMENT ENDS. Anything written after this will not appear on the pdf.