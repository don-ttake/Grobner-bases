\documentclass[11pt, oneside]{article}   	
\usepackage[margin=1in]{geometry}                		
\geometry{letterpaper}                   		
\usepackage{graphicx}						
\usepackage{amssymb}
\usepackage{amsmath}
\usepackage{amsthm}
\usepackage{enumerate}

%Commands above this line set up the type of document, and ensure it has access to the LaTeX files needed to understand your commands.

%Here, I define some "shortcuts" for notation I commonly use. 
\newcommand{\N}{\mathbb N}
\newcommand{\Z}{\mathbb Z}
\newcommand{\Q}{\mathbb Q}
\newcommand{\C}{\mathbb C}
\newcommand{\R}{\mathbb R}
\newcommand{\F}{\mathbb F}

\newtheorem*{proposition}{Proposition}
\newtheorem*{theorem}{Theorem}

\title{Homework 1}
\author{The Author} 
%\date{}			




%THIS IS WHERE THE ACTUAL TEXT OF THE DOCUMENT BEGINS.				
\begin{document}

\begin{center}\noindent{\bf Math 333:  Homework \#1}\\Mingchen Li\\ \end{center}

\hrulefill %This just makes a nice horizontal line. Useful if you like to separate your problems with lines!

Note: All problems are from Gallian unless otherwise noted. The abbreviation ``DF" indicates that the problem comes from Dummit \& Foote. Problems marked with a $*$ are extra credit and are optional.

\begin{enumerate}

\item Prove that the dihedral group $D_{n}$ will always have 2n elements.  Here, $D_n$ is the group of symmetries of a regular $n$-gon (also known as the group of rigid motions on the $n$-gon).  [Hint: do not overcomplicate things. You do not need to resort to matrix representations or generators to prove this result. A simple counting argument will suffice here! ]

\begin{proof}
Let shape A to be a regular $n$-gon where $n \in \N$ is arbitrary. Denote its vertices clockwise by $v_1, v_2 ...v_n$. Given that the dihedral group is the set of symmetries on the polygon, denote the dihedral group generated by A by G.

\newline We shall begin our counting by doing the rotation first. Since rotating a regular $n$-gon by 360/n degree yields a symmetric copy of the original shape. Thus by rotating n times we yields n symmetries of the shape A. Thus we have found n distinct elements in group G.

 \linebreak For each vertex $v_j$, $j\in [1,n]$, it has two fixed neighbors $v_{j-1}$ and $v_{j+1}$. When we number the vertices by clockwise order, the vertex with greater number is on the relative right side, with up pointing outside the shape. However after performing a reflection on one of the symmetry axes (regular n-gons have n symmetric axes), the vertex numbering becomes counter-clockwise with bigger numbering on the left. 
 
 \linebreak Then we proceed to perform the rotation method again. Since the numbering is counter clockwise, each rotation copy is different from the clockwise copy. We have hereby obtained another n elements in G. Denote our collection of 2n elements as $\hat{G}$. We have shown $\hat{G}\subseteq G$.
 
 Now we aim to prove $\hat{G}\supseteq G$. For arbitrary element in G, call it g, it is an symmetry of the original shape A. Thus we can find its numbering on each vertex. If the numbering is in clock-wise, then it is one of the rotation symmetry without performing reflection. If it is counter clockwise, then it is one of the rotation symmetry after performing one reflection. Thus $g\in \hat{G}$, $\hat{G}= G$, G has 2n elements.
  
%Type your solution here!
\end{proof}

\item[{\bf 2.22}:] (Left-Right cancellation implies commutativity). Let $(G,*)$ be a group with the property that {\it for any} $x,y,z$ in the group, if $x*y=z*x$, then $y=z$. Prove that $G$ is an Abelian group. 
\begin{proof}
Let x, y to be arbitrary elements in G, we aim to show
\[ x * y = y*x \]
Consider the equation below
\[x*y*x=x*y*x\]
Since $x,y\in G$, the operation * is associative and closed. We know
\[x*(y*x)=(x*y)*x\] 
Where $(y*x),(x*y)\in G$. Then by the property given, ie. $X*Y=Z*X \Rightarrow{}Y=Z$ 
\newline Let $Y=(y*x), Z=(x*y)$, we have $y*x=x*y$.


\end{proof}
%Type your solution here!

\item[{\bf Example 11}] For $n>1$, let $U(n)$ denote the set of positive integers less than $n$ and relatively prime to $n$. Show that $U(n)$ forms a group under multiplication modulo $n$, and construct the Cayley table for $U(12)$. (You may use number theory facts about divisibility and relative primes without proof, but you must state any facts you use. )
\begin{proof}
We shall denote gcd(a,b) to be the greatest common divisor of a and b. For arbitrary $x\in \N$, we generate $U(x)$ to be the set of integers that are relatively prime to x. To show $U(x)$ to be a group under multiplication modulo n, it must satisfy following properties:
\begin{enumerate}
    \item closed under multiplication modulo n: Assume that there exist some $a, b \in U(x)$ for some $x \in \N$ such that $a \times b$ mod $12 = c \notin U(x)$. This implies that $gcd(c,x)!=1$, ie. $x=k\times c$ for some $k\in \N, k<x$. Thus:
    \[a \times b =k\times c \times z + c = (z\times k +1)\times c \]
    For some $z\in \N$. Given the property above, either a or b is a multiple of c, which make gcd(a,x) or gcd(b,x) to be at least $c>1\Rightarrow\Leftarrow$
    \item Exists identity element: 1 is the identity element. Since for $\forall x\in \N$, $gcd(x, 1)= 1 \Rightarrow{} 1\in U(x)$. At the same time, by basic arithmetic, $1\times n = n \times 1= n$ for $\forall n \in \N$
    \item Exists inverse: I can only prove this with my knowledge of Extended Euclidean algorithm that I learned in MA398 Cryptography. With Extended Euclidean algorithm one can find x,y given a, b in the following equation:
\[ax+by=gcd(a,b)\]
    In this case we can make $a\in U(n), b=n$. Since $gcd(a, n)=1$, the equation becomes:
\[ax+bn=1\]
    Which is equivalent to $ax=1$ mod n. We have found one-sided inverse of a being x. Since multiplication is commutative, x is the two-sided inverse of a. 
    \item Associative: Multiplication on N is associative. For any $a, b, c \in U(x)$, $(a\times b) \times c = a\times (b \times c)$. At the same time: ((a mod x) $\times$ b) mod x= a$\times$ b mod x base on previous knowledge. We have concluded that the opertation is associative.
\end{enumerate}
%Type your solution here!
\end{proof}

%U(12) in graph


\begin{center}
 \begin{tabular}{||c c c c c||} 
 \hline
  & 1 & 5& 7 & 11  \\ [0.5ex] 
 \hline\hline
 1 & 1 & 5 & 7 & 11 \\ 
 \hline
 5 & 5 & 1 & 11 & 7 \\
 \hline
 7 & 7 & 11 & 1 & 5  \\
 \hline
 11 & 11 & 7 & 5 & 1 \\
 \hline
 
\end{tabular}
\end{center}


\item[{\bf DF 1.1.8}] Let $G=\{z\in \C \, | \,z^n=1 \text{ for some } n\in \N\}$.
\begin{enumerate}
\item[a)] Prove that $G$ is a group under multiplication (called the group of {\it roots of unity} in $\C$).
\begin{proof}To show G is a group under multiplication, it should satisfy following properties:
\begin{enumerate}
    \item Closed under multiplication: For arbitrary $a,b \in G$, $a^m = b^n = 1$  for some $m, n \in \N$. $(a\times b)^{mn}= (a^m)^n\times (b^n)^m=1$. Thus $a\times b\in G$.
    \item Existence of identity: 1 is the identity in G. Since $1^n=1, n\in \N\Rightarrow{} 1\in G$. $1\times a=a\times 1=a$ for arbitrary $a\in G$. We have found our identity.
    \item Existence of inverse element. Let $a\in G$ to be arbitrary element in G. It can be written as $a=re^{i\theta}$. Given that there is some $k\in \N$ such that:
    \[a^k=r^ke^{ki\theta}=1\]
    Equivalently we can define $b=\dfrac{1}{r}e^{-i\theta}$. It is obvious that $a\times b=b\times a=1$. At the same time, $(a\times b)^k=a^k \times b^k = 1 \times b^k = 1$. Thus $b\in G$ and we have found our inverse.
    \item Associative: Base on basic arithmetic property, multiplication is associative. 
\end{enumerate}
Base on above properties, G is hereby a group.
\end{proof}
\item[b)] Prove that $G$ is not a group under addition.\\
\begin{proof}
This can be proven with a counter example:
We know $1,-1\in G$ since, $1^1=1, (-1^2)=1$. Assume G is a group under addition. Then it must be closed under addition. However $1-1=0$ and $0\notin G$ since $\forall n\in \N, 0^n=0$.

\end{proof}
\end{enumerate}  


\item[{\bf 2.37*:}] For five extra-credit points: Let $G$ be a finite group. Show that the number of elements $x$ of $G$ such that $x^3=e$ is odd. Show that the number of elements $x$ of $G$ such that $x^2\neq e$ is even.\\

\begin{enumerate}
    \item It is obvious that $e^3=e$. For every other element $a\in G$, if $a^3=e$, then by basic property of group, $(a^{-1})^3=e $ as well. Thus aside from $e$, every element in the group that satisfy the equation must have a pair element that also satisfy the equation. Along with the e itself, we have odd numbers of answers. 
    \item Assume $\exists a\in G$ such that $a^2\neq e$. Then its inverse must have the same property as well. This can be proven by simple contradiction. If $(a^{-1})^2=e$, then 
    \[(aa^{-1})^2=e^2=e\neq a^2=a^2(a^{-1})^2=(aa^{-1})^2 \Rightarrow\Leftarrow\]
    Thus the set of answers must exist in pair. ie. is even.
\end{enumerate}



\end{enumerate}

\end{document}  

%THIS IS WHERE THE DOCUMENT ENDS. Anything written after this will not appear on the pdf.