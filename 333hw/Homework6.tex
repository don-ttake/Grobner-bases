\documentclass[11pt, oneside]{article}   	
\usepackage[margin=1in]{geometry}                		
\geometry{letterpaper}                   		
\usepackage{graphicx}						
\usepackage{amssymb}
\usepackage{amsmath}
\usepackage{amsthm}
\usepackage{enumerate}

%Commands above this line set up the type of document, and ensure it has access to the LaTeX files needed to understand your commands.

%Here, I define some "shortcuts" for notation I commonly use.
\newcommand{\N}{\mathbb N}
\newcommand{\Z}{\mathbb Z}
\newcommand{\Q}{\mathbb Q}
\newcommand{\C}{\mathbb C}
\newcommand{\R}{\mathbb R}
\newcommand{\F}{\mathbb F}

\newcommand{\stab}{\operatorname{stab}}
\newcommand{\orb}{\operatorname{orb}}
\newcommand{\Aut}{\operatorname{Aut}}
\newcommand{\Inn}{\operatorname{Inn}}

\newtheorem*{proposition}{Proposition}
\newtheorem*{theorem}{Theorem}

\title{Homework 5}
\author{The Author}
%\date{}			




%THIS IS WHERE THE ACTUAL TEXT OF THE DOCUMENT BEGINS.				
\begin{document}

\begin{center}\noindent{\bf Math 333:  Homework \#6}\\Mingchen Li\\ \end{center}
\thispagestyle{empty}



\hrulefill %This just makes a nice horizontal line. Useful if you like to separate your problems with lines!

In the first four problems, you will prove some results to related to the Fundamental Theorem of Finite Abelian Groups.



\begin{enumerate}



\item[{\bf Problem 1:}] Let $m$ and $n$ be relatively prime positive integers. Prove that $\Z/m\Z\times \Z/n\Z\cong \Z/(mn)\Z$.
\begin{proof}
Consider the mapping $\varphi:(a,b)\mapsto ab$ mod $mn$. We aim to show that this is an isomorphism. 
\begin{enumerate}
    \item[Bijection: ] Take arbitrary $x\in \Z/(mn)\Z$, by the Fundamental theorem of Arithmetic, $x= x_1x_2..x_k$ where each $x_i$ is distinct prime. Since $gcd(m,n)=1$, $m, n$ do not share any prime divisor. Thus for all $x_i$ it is either a divisor or $m$ or $n$. We can then rearrange the expression of $x$ to this:
    \[x=x_m_1x_m_2...x_m_ix_n_1x_n_2...x_n_j\]
    Where $x_m_i$s are prime divisor of $x$ as well as $m$. Then consider $x_m=\prod_{a=1}^{i}x_m_a$, and $x_n$ in similar definition. $x_m\in \Z/m\Z$ and similarly $x_n\in \Z/n\Z$. Thus $\varphi(x_m, x_n)= x$. $\varphi$ is surjective.
    
    If there are some $(x_m, x_n)$, $(y_m, y_n)$ such that $\varphi(x_m, x_n)= x =\varphi(y_m, y_n)$. Then using the same setting above, $x$ can be break into two parts and $x_m=x_m_1x_m_2...x_m_i=y_m$. Since multiplication is associative and commutative, $x_m =y_m$ and the same relationship holds for $x_n$ and $y_n$. Thus $\varphi$ is injective.
    
    \item[Preserve operation: ] Let $(x_m, x_n)$, $(y_m, y_n)$ be arbitrary elements in $\Z/m\Z\times \Z/n\Z$. We aim to show $\varphi(x_my_m, x_ny_n)=\varphi(x_m, x_n)\varphi(y_m, y_n)$
    
    We know $\varphi(x_my_m, x_ny_n)=x_my_mx_ny_n$ mod $mn$ 
    
    $\varphi(x_m, x_n)\varphi(y_m, y_n)= x_mx_n\text{ mod } mn \times_{mn} y_my_n \text{mod } mn = x_mx_ny_my_n \text{ mod } mn$
    
    Since modular multiplication is associative, the mapping hereby preserves operation.
\end{enumerate}
\end{proof}

\newpage
\item[{\bf Problem 2:}] \begin{itemize}
\item  Let $G$ be an Abelian group. Show that for any $n\in \N$, the set $\{x\,|\, x^n=e\}$ forms a subgroup of $G$. 
\item Let $G$ be an abelian group, and $n\in \N$. Show that $G^n = \{x^n\,|\, x\in G\}$ forms a subgroup of $G$.
\end{itemize}

\begin{itemize}
    \item[proof for part 1:] Let $n\in \N$ be arbitrary, we aim to use two step strategy to show that $H=\{x\,|\, x^n=e\}$ forms a subgroup of $G$.
    
    \begin{enumerate}
        \item[not empty:] We know $e^n=e$, so $e\in H$, and H is not empty.
        \item[closed: ] take arbitrary $a,b \in H$, $a^n=b^n=e$. Since $G$ is an abelian group, $(ab)^n=abababab...ab= aa...abbb..b=a^nb^n=ee=e$. Thus it is closed.
        \item[exists inverse:] take arbitrary $a\in H$, we know $a^n=aa^{n-1}=e$. At the same time, $(a^{n-1})^n= (a^{n})^{n-1}=e^{n-1}=e$. Thus $a$ has inverse in $H$ being $a^{n-1}$.
        
    \end{enumerate}
    \item[proof for part 2:] Let $n\in \N$ be arbitrary, we aim to use two step strategy to show that $H=\{x^n\,|\, x\in G\}$ forms a subgroup of $G$.
    \begin{enumerate}
        \item[not empty:] We know $e\in G$ regardless what size of G is. And $e^n=e$, thus $e\in H$
        \item[closed: ]Take arbitrary $a^n, b^n\in H$, Since G is a group, it is hereby closed. $ab\in G$. By the defition of $H$, we know $(ab)^n\in H$ 
        
        At the same time, since $G$ is abelian, $a^nb^n=aa...abbb..b=ababab...ab=(ab)^n\in H$
        
        \item[exists inverse: ] Take arbitrary $a^n\in H$, we know that $a$ has inverse $a^{-1}\in G$. By definition of group H, we know $(a^{-1})^n=a^{-n}\in H$. Since G is an abelian group, we have
        
        $a^na^{-n}=aa...aa^{-1}a^{-1}..a^{-1}=aa^{-1}aa^{-1}...aa^{-1}=e^n=e$. Thus $a^{-n}$ is the inverse of $a^n$. 
    \end{enumerate}
\end{itemize}


\newpage
\item[{\bf Problem 3:}] List (with justification) all non-isomorphic Abelian groups...
\begin{itemize}
\item ...of order 30
\item ...of order 27
\item ...of order 24
\end{itemize}
By the Fundamental Theorem of Finite Abelian Group

every finite Abelian group $G$ is isomorphic to a direct product of the form:
\[\Z/p_1^{n_1}\Z \times \Z/p_2^{n_2}\Z...\times\Z/p_k^{n_k}\Z\]
Where $|G|=\prod_{i=1}^{k}p_i^{n_i}$, $p_i$s are not necessarily distinct, $n\geq 1$, $n\in \N$. 

Moreover, the number of terms in the product and the orders of the cyclic groups are \textbf{uniquely} determined by the group.

The uniqueness of partition guarantees that each two distinct partitions of group $G$ yields distinct, isomorphism classes. With this knowledge, we can break the finite Abelian groups into different distinct isomorphism classes.

\begin{enumerate}
    \item[$|G|$= 30:] \begin{itemize}
        \item $\Z/2\Z \times \Z/3\Z\times\Z/5\Z$
    \end{itemize}
    \item[$|G|$= 27:] \begin{itemize}
        \item $\Z/3\Z \times \Z/3\Z\times\Z/3\Z$
        \item $\Z/9\Z \times \Z/3\Z$
        \item $\Z/27\Z$
    \end{itemize}
    \item[$|G|$= 24:]\begin{itemize}
        \item $\Z/2\Z \times \Z/2\Z\times\Z/2\Z\times\Z/3\Z$
        \item $ \Z/4\Z\times\Z/2\Z\times\Z/3\Z$
        \item $ \Z/8\Z\times\Z/3\Z$

    \end{itemize}
\end{enumerate}

\newpage
\item[{\bf Problem 4:}] Complete the statement, and prove your result: \\


{\it There exists exactly one Abelian group of order $n$ (up to isomorphism) if and only if $n=p_1p_2p_3..p_k$ where each $p_i$ is distinct prime }


\hrulefill

\begin{proof}
Assume that it is not the case. That there is an unique Abelian group $G$ of order $n$ where $n=p_1^{r_1}p_2^{r_2}p_3^{r_3}..p_k^{r_k}$ and at least one of the $r_i$s is greater than 1. Without loss of generality, let $r_1=r_{1a}+r_{1b}$ where $r_{1a}, r_{1b}\geq 1$. Then by the Fundamental Theorem of Finite Abelian Group, there are at least two distinct isomorphism class of order $n$: \begin{itemize}
    \item $\Z/p_1^{r_{1a}}\Z \times \Z/p_1^{r_{1b}}\Z...\times\Z/p_k^{r_k}\Z$
    \item $\Z/p_1^{r_1}\Z \times ...\times\Z/p_k^{n_k}\Z$
\end{itemize}
Which arrives to a contradiction. 
\end{proof}

\newpage
\item[{\bf Problem 5:}] 
\begin{itemize}
\item Let $G$ be a group, and define the set of {\it automorphisms} of $G$ to be the set $$\Aut(G) = \{\varphi:G\rightarrow G\,|\, \varphi \text{ is an isomorphism }\}$$
Prove that $\Aut(G)$ forms a group under composition.
\begin{proof}
To show $\Aut(G)$ forms a group, we aim to show the following properties: \begin{itemize}
    \item[Exists identity:] Consider the identity map e :$G\rightarrow G$. Given an arbitrary $\varphi\in \Aut(G)$, we know $\varphi \circ e= e \circ \varphi =\varphi$ by function composition. At the same time, e is bijection and $e(ab)=ab=e(a)e(b)$ for $a,b\in G$. Thus e is an isomorphism.
    \item[Exists inverse:] Consider an arbitrary $\varphi \in \Aut(G)$, since $\varphi$ is bijection, it has an inverse function $\varphi^{-1}: G\rightarrow G$. And $\varphi\varphi^{-1}=\varphi^{-1}\varphi=e$.
    
    Then we aim to show $\varphi^{-1}$ is an isomorphism. For arbitrary $a,b \in G$, we know there are $x,y\in G$ such that $\varphi(x)=a$, $\varphi(y)=b$. Since $\varphi$ is isomorphism, we know $$\varphi^{-1}(ab)=\varphi^{-1}(\varphi(xy))=xy$$
    $$\varphi^{-1}(a)\varphi^{-1}(b)=\varphi^{-1}(\varphi(x))\varphi^{-1}(\varphi(y))=xy$$
    Thus we have found the inverse. 
    \item[Closed:] Take arbitrary $\varphi_1, \varphi_2\in \Aut(G)$, since they are both isomorphism, we have:
    \[\varphi_1\circ \varphi_2(ab)=\varphi_1(\varphi_2(a)\varphi_2(b))=\varphi_1(\varphi_2(a))\varphi_1(\varphi_2(b))= \varphi_1\circ \varphi_2(a)\varphi_1\circ \varphi_2(b)\]
    Thus it is closed.
    \item[Associative:] 
\end{itemize}
\end{proof}
\item Determine\footnote{This means that you need to determine what (known) group $\Aut(G)$ is isomorphic to. It may help to note that both $\Z$ and $\Z/n\Z$ are cyclic groups.} $\Aut(\Z)$, where $\Z$ is the group of integers under addition.  Then, determine $\Aut(\Z/n\Z)$ for any integer $n$.
\begin{proof}
Since $\Z$ is generated by either 1 or -1, for arbitrary $\varphi \in \Aut(G)$, it have to map generators to generators to be surjective. Thus it is crucial to determine which of the two generator does the generators of $\Z$ is mapped to. For example, if $\varphi(1)=n$ and $|n|\neq 1$, then the image of the map will not be full $\Z$ but only the multiples of $n$. Thus there are essentially two classes of isomorphisms: $\varphi_1(1)=1$ and $\varphi_2(1)=-1$. Thus $\Aut(\Z)\cong \Z/2\Z$.

In similar vein, the $\Aut(\Z/n\Z)$ must map generator to generators. Since the map target of $\varphi(1)$ will determine the rest of the elements' mapping, 1 must map to generator of $\Z/n\Z$. According to earlier knowledge, we know an element of $\Z/n\Z$ has order n if it is relatively prime to n. Thus there are $\phi(n)$ possible isomorphic classes in $\Aut(\Z/n\Z)$ where $\phi$ is Euler phi function. $\Aut(\Z/n\Z) \cong \Z/\phi( n)\Z)$
\end{proof}
\end{itemize}


\newpage
\item[{\bf Problem 6:}] 
\begin{itemize}
\item Let $G$ be a group, and $a\in G$. Define the function $\phi_a:G\rightarrow G$ by $\phi_a(x) = axa^{-1}$. Prove that $\phi_a$ is an automorphism of $G$ (called the {\it inner automorphism of $G$ induced by $a$}).
\item Prove that the set of inner automorphisms, denoted $\Inn(G)$ of a group $G$ forms a subgroup of $\Aut(G)$.
\end{itemize}
\begin{itemize}
    \item[proof for part 1: ] We aim to show that $\phi_a$ is an isomorphism. We shall begin to show that it is homomorphism. 
    
    Take arbitrary $x, y\in G$, $\phi_a(xy)=axya^{-1}=axa^{-1}aya^{-1}=\phi_(x)\phi_a(y)$
    
    Take arbitrary $x, y \in G$, if $\var_a(x)=\var_a(y)$, then $axa^{-1}=aya^{-1}$ and $x=y$ by canceling the terms. Thus $\var_a$ is injective.
    
    For arbitrary $g\in G$, $a^{-1}ga\in G$ since $G$ is closed. Then $\var_a(a^{-1}ga)=g$. Thus $\var_a$ can map to any $g\in G$ hereby Surjective. 
    \item[proof for part 2:]
    Since identity mapping is an inner automorphism, denoted as $\phi_e(x)=exe^{-1}=x$, $\Inn(G)$ is not empty. 
    
    Let $\phi_a, \phi_b\in \Inn(G)$ be arbitrary. Then $(\phi_b)^{-1}(x)=b^{-1}(x)b$ since \[(\phi_b)^{-1}(\phi_b)(x)=(\phi_b)(\phi_b)^{-1}(x)=bb^{-1}xb^{-1}b=x\]
    Thus by the one step subgroup test, we have: \[\phi_a(\phi_b)^{-1}(x)=ab^{-1}xba^{-1}=\phi_{ab^{-1}}(x)\in \Inn(G)\]
    
\end{itemize}
\newpage
\item[{\bf Problem 7:}] Recall that for any group $G$, $Z(G)$ denotes the center of $G$ (the elements which commute with all other elements). Prove that $G/Z(G)\cong \Inn(G)$ for any group $G$.
\begin{proof}
Consider the mapping: $\Phi(g)=\phi_g $ for $g\in G$. We aim to show it satisfy the condition of First Isomorphism Theorem:

For arbitrary $\phi_g \in \Inn (G)$, we know $\Phi(g)$ maps to this automorphism. Thus $Phi$ is surjective.

For arbitrary $g, h\in G$, $\Phi(gh)=\phi(gh)$. Since $\phi$ is automorphism, it preserves operation. $=\phi(g)\phi(h)=\Phi(g)\Phi(h)$. Thus $\Phi$ is an homomorphism. 

If $g\in Z(G)$, by commutivity: $\Phi(g)=\phi_g=gxg^{-1}=gg^{-1}x=x$. Thus $g\in Ker(\Phi)$. If $g\in  Ker(\Phi)$, then $\Phi(g)=$ identity function. Then $gxg^{-1}=x$ for arbitrary $x\in G$. So $gx=xg$, meaning g commutes with arbitrary element in $G$, $g\in Z(G)$. $Ker(\Phi)=Z(G)$.

Thus we can use the First isomorphism theorem and conclude that $G/Z(G)\cong \Inn(G)$ for any group $G$.

\end{proof}
\newpage
\item[{\bf Problem 8:}] Using the previous results, and a result from previous homework, prove that if $G$ is non-Abelian, then $\Aut(G)$ is not cyclic. Give an example to show that the converse of this theorem doesn't hold (i.e., there are Abelian groups such that $\Aut(G)$ is not cyclic.)

\begin{proof}
Given G is not abelian, suppose that $\Aut(G)$ is cyclic. Then since $\Inn(G)$ is a subgroup of $\Aut(G)$, $\Inn(G)$ is cyclic as well. However, by problem 7 we know $G/Z(G)\cong \Inn(G)$. Since isomorphism preserves cyclic property, we know $G/Z(G)$ is also cyclic. By the $G/Z$ theorem on Gallian: if $G/Z(G)$ is cyclic, then $G$ is abelian. Thus we have arrived at a contradiction.
\end{proof}

\end{enumerate}

\end{document}  

%THIS IS WHERE THE DOCUMENT ENDS. Anything written after this will not appear on the pdf.