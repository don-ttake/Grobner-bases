\documentclass[11pt, oneside]{article}   	
\usepackage[margin=1in]{geometry}                		
\geometry{letterpaper}                   		
\usepackage{graphicx}						
\usepackage{amssymb}
\usepackage{amsmath}
\usepackage{amsthm}
\usepackage{enumerate}

%Commands above this line set up the type of document, and ensure it has access to the LaTeX files needed to understand your commands.

%Here, I define some "shortcuts" for notation I commonly use.
\newcommand{\N}{\mathbb N}
\newcommand{\Z}{\mathbb Z}
\newcommand{\Q}{\mathbb Q}
\newcommand{\C}{\mathbb C}
\newcommand{\R}{\mathbb R}
\newcommand{\F}{\mathbb F}

\newcommand{\stab}{\operatorname{stab}}
\newcommand{\orb}{\operatorname{orb}}
\newcommand{\Aut}{\operatorname{Aut}}
\newcommand{\Inn}{\operatorname{Inn}}

\newtheorem*{proposition}{Proposition}
\newtheorem*{theorem}{Theorem}

\title{Midterm Review}
\author{The Author}
%\date{}			




%THIS IS WHERE THE ACTUAL TEXT OF THE DOCUMENT BEGINS.				
\begin{document}

\begin{center}\noindent{\bf Math 333:  Midterm Review}\\Nora Youngs\\ \end{center}
\thispagestyle{empty}
$Z_n={0,..n-1}$ under modulo addition


This document provides a list of the material to study for the first midterm. In general, you need not perfectly memorize definitions/theorems word-for-word - it's much more important that you understand the general meaning, any important properties, and can use the objects and theorems properly. 

Some questions to ponder as you study, to help think about the purpose of the material we've learned:

\begin{enumerate}
\item What is the use of showing something is a group? Why would it be advantageous to know that a set had this property?
\item What is the purpose of thinking about ``isomorphism" when you're working with groups?
\item Homomorphism isn't as strong as isomorphism, so why do we bother to define the idea?
\item What kinds of groups are ``nicest?"  If you had to work with any kind of group for a computational purpose, which one would be the easiest to handle? 
\item One of the properties a subgroup may (or may not) have is normalcy. What is the benefit of knowing that a subgroup is normal?
\end{enumerate}
\hrulefill %This just makes a nice horizontal line. Useful if you like to separate your problems with lines!








\newpage
\begin{enumerate}
\item Definitions to know:

\begin{itemize}
\item group: associative, exist inverse, closed, exist identity
\item subgroup: not empty, exist inverse and closed or $ab^{-1}\in G$ and not empty
\item coset: $H\leq G$. the left coset of H in G is the set $\{ah|a\in G, h\in H\}$ denoted $aH$. some properties: \begin{enumerate}
    \item $aH=H $ iff $a\in H$
    \item $(ab)H=a(bH)$
    \item either $aH=bH$ or $aH\cap bH=\varnothing$
    \item $aH=bH\Longleftrightarrow a\in bH \Longleftrightarrow a^{-1}b \in H$
    \item $aH\leq G$ iff $a\in H$
\end{enumerate}
\item normal subgroup:  a subgroup is normal if $aH=Ha$ for all a $\in G$
\item quotient group: the set of cosets, 

Given G be a group and H be a normal subgroup of G, the set $G/H=\{aH|a\in G\}$ is a group under the operation of $(aH)(bH)=(ab)H$
\item abelian: commutative
\item cyclic: $\langle a \rangle$
\item order (group and element)
\item center: the set of elements that commutes with every element in G

Let G be a group and Z(G) be center, if $G/Z(G)$ is cyclic, then G is abelian.

$G/Z(G)$
\item centralizer: $C(a)=\{g\in G|ag=ga\}$ It's a subgroup of G.
\item orbit: places G can carry i to. G is a group of permutations on S, $orb_G(i)=\{\varphi (i)|\varphi\in G\}$
\item stabilizer: permutations that fixes i. $stab_G(i)=\{\varphi|\varphi(i)=i, \varphi \in G\}$
\item homomorphism
\item isomorphism
\item kernel
\end{itemize}








\newpage
\item Examples we have investigated:
\begin{itemize}
\item $\Z$, $\Q$, $\R$, $\C$ under addition
\item $\Q^*$, $\R^*$, $\C^*$, $\Q_{>0}$ under multiplication (and the unit-circle subgroup of $\C^*$, often denoted $S^1$.)
\item $\Z/n\Z$ under modular addition
\item $U(n)$ under modular multiplication
\item Matrix groups under multiplication, especially $GL_n(\R)$, $SL_n(\R)$
\item cyclic groups
\item permutation groups (symmetric $S_n$, alternating, even permutation $A_n$, etc)
\item dihedral groups
\item rotation groups
\item direct products of groups
\item quotient groups
\end{itemize}



\newpage

\item Theorems to be able to use (and state fairly precisely:)
\begin{itemize}
\item Lagrange's theorem: subgroup order divide order of parent group. 
\item The Fundamental Theorem of Cyclic groups: Every subgroup of a cyclic group is cyclic. Moreover, any subgroup of $\langle a \rangle$'s order divides $|\langle a \rangle|=n$. For each positive divisor k of n, there is exactly on subgroup of order k: $\langle a^{n/k} \rangle$

important corollaries: \begin{enumerate}
    \item in a finite group, the number of elements of order d is a multiple of $\phi(d)$, (coprimes less than d)
\end{enumerate}
\item The First Isomorphism Theorem
\item The Fundamental Theorem of Finite Abelian Groups

Every finite Abelian group is a direct product of cyclic groups of prime-power order. Moreover, the number of terms in the product and the orders of the cyclic groups are uniquely determined by the group.

\item The orbit-stabilizer theorem: \[\text{G is finite group of permutations of a set S, then for any } i\in S: |G|=|orb_G(i)||stab_G(i)|\]
\item Cauchy's theorem
\end{itemize}

\newpage
\item Methods: you should be able to show
\begin{itemize}
\item  $G$ with a certain operation forms a group (or doesn't!)
\item $G$ is abelian (or is not)
\item $G$ is cyclic (or is not)
\item $H$ is a subgroup of $G$ (or isn't)
\item $H$ is (or is not) a normal subgroup of $G$:

Show $aHa^{-1}\subseteq H$ 
\item $\phi:G\rightarrow A$ is (or isn't) a homomorphism/isomorphism
\item $G_1$, $G_2$ are isomorphic (or aren't).
\end{itemize}

\newpage


\item Theorems/results I might ask you to prove on the exam. In all cases, if the proofs use other theorems or lemmas in a minor way, you will not be required to prove those results. Feel free to ask me if you are not sure what would be required! {\bf I will ask you one or two of these; they will not be the only questions on the exam.}
\begin{enumerate}


\item Lagrange's theorem: $|H|$ divides $|G|$ and the number of distinct left costs of H in G is $|G|/|H|$. 

The key to the proof is to write $G=a_1H\cup ..a_rH$ and they are all disjoint and equal to $|H|$

Some IMPORTANT corollaries:\begin{enumerate}
    \item $|a| \text{ divides } |G|$
    \item group of prime order are cyclic since $|\langle a \rangle|$ divides prime
    \item $a^{|G|}=e$
    \item Fermat's little theorem: a is int p is prime, $a^p\mod p= a \mod p$
\end{enumerate}


\item The First Isomorphism Theorem: $\phi$ is a group homomorphism form G to $\bar{G}$. Then the maping from $G/Ker\phi$ to $\phi(G)$ given by $gKer\phi \longrightarrow \phi(g)$ is an isomorphism. 
\begin{proof}
well-defined (that is, the correspondence is independent of the particular coset representative chosen) and one-to-one follows directly from property 5 of Theorem 10.1: $\phi(a)=\phi(b)$ iff $aKer\phi=bKer\phi$. Show its order preserving by using the order preserving of $\phi$
\end{proof}
\item Cayley's theorem (any group is isomorphic to a subgroup of a permutation group)
\begin{proof}
Aim to find a group of permutations that G is isomorphic to, for $g\in G$,define 
\[T_g(x)=gx \text{ for all }  x \in G\]
Show this is a permutation. Then show
\[\bar{G}=\{T_g|g\in G\}\]
is a group, then use $\phi(g)=T_g$ to create isomorphism 
\end{proof}


\item Cauchy's theorem (any finite abelian group $G$ has an element of order $p$ for any prime which divides $|G|$).
\begin{proof}
Use induction on order of G, handle G with order 2. Then assume true when G has order less than p, show it's the case on p.

find an element of prime order: if $|x|=m=qn$ then $|x^n|=q$. If p=q, we are done. 

If not, since every subgroup of abelian group is normal, For arbitrary $x,y \in \bar{G}$, $x=g_a\langle x\rangle, y=g_b\langle x\rangle$ for some $g_a, g_b\in G$. Since G is abelian, x, y are hereby normal:

\[xy= g_a\langle x\rangle g_b\langle x\rangle=g_a g_b\langle x\rangle= g_bg_a\langle x\rangle =g_b\langle x\rangle g_a\langle x\rangle=yx\]

Thus $G'$ is abelian, it can be applied to our induction assumption. By Lagrange's theorem:
\[|G/\langle x\rangle||\langle x\rangle|=|G|\]

Since $|G|=mp, |\langle x\rangle|=q$, $p$ divides $|G'|$ and by induction assumption, there is an element in $G'$, denoted as $y\langle x\rangle$, that has order of p. Thus:
\[(y\langle x\rangle)^p=y^p\langle x\rangle=\langle x\rangle\]
Thus $y^p\in\langle x\rangle$. Either $y^p=e$, or $y^p$ has order q and we can find $y^q$ to be the element in G that have order p.
\end{proof}


\item $G/Z(G)$ is cyclic if and only if $G$ is abelian.
\begin{proof}


$\Rightarrow:$ 
Abelian $\Longleftrightarrow$ $Z(G)=G$. So just show the only element in $G/Z(G)$ is $Z(G)$

Every $a\in G$, $aZ(G)=g^n Z(G)$. Thus $a=g^nz$ for some $z\in Z(G)$. Since $g_n$ and $z$ commutes with $g$. i.e., in $C(g)$, closed, then $a\in C(g)$ as well. Thus $g\in Z(G)$ as well, $gZ(G)=Z(G)$

$\Leftarrow$ Show $Z(G)=G$, show $G/Z(G)=\{eZ(G)\}$
\end{proof}


\item $\Q$ is not cyclic


\item If $H\leq S_n$, then either every member of $H$ is even, or exactly half of the members of $H$ are even.

\begin{proof}
Since H is subgroup, every element must have inverse. The inverse share the same type as the original since identity is even. 

It is trivial if all are even.

If there are both even and odd elements, show $\{odd\}=\{even\}$ by composing odd with odd to show $\subseteq$ and compose even with odd to show $\supseteq$
\end{proof}
\item Determination of $\Aut(\Z/n\Z)$ (see HW 6)

map generator to generator of $\Z/n\Z$
\end{enumerate}
\end{enumerate}




\newpage
\hrulefill


Classification related:
\begin{enumerate}
    \item Classification of Group of Order p: $$\text{they are cyclic and this results from Lagrange}$$
    \item Classification of Groups of Order 2p: $$\text{G is a group of order 2p and p is greater than 2 prime. } G \cong \Z_{2p} or $$
    \item classification of Group of order 4: a group of order 4 $\cong \Z_4$ or $\Z_2\times \Z_2$ 
    \item Classification of Groups of Order $p^2$: is isomorphic to either $\Z_{p^2} $ or $\Z_p \times \Z_p$. This also shows the group to be Abelian
    \item cyclic group are abelian but not other way around.
    \item Every finite Abelian group is a direct product of cyclic groups of prime-power order. Moreover, the number of terms in the product and the orders of the cyclic groups are uniquely determined by the group.
\end{enumerate}
\end{document}  

%THIS IS WHERE THE DOCUMENT ENDS. Anything written after this will not appear on the pdf.