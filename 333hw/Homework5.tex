\documentclass[11pt, oneside]{article}   	
\usepackage[margin=1in]{geometry}                		
\geometry{letterpaper}                   		
\usepackage{graphicx}						
\usepackage{amssymb}
\usepackage{amsmath}
\usepackage{amsthm}
\usepackage{enumerate}

%Commands above this line set up the type of document, and ensure it has access to the LaTeX files needed to understand your commands.

%Here, I define some "shortcuts" for notation I commonly use.
\newcommand{\N}{\mathbb N}
\newcommand{\Z}{\mathbb Z}
\newcommand{\Q}{\mathbb Q}
\newcommand{\C}{\mathbb C}
\newcommand{\R}{\mathbb R}
\newcommand{\F}{\mathbb F}

\newcommand{\stab}{\operatorname{stab}}
\newcommand{\orb}{\operatorname{orb}}

\newtheorem*{proposition}{Proposition}
\newtheorem*{theorem}{Theorem}

\title{Homework 5}
\author{The Author}
%\date{}			




%THIS IS WHERE THE ACTUAL TEXT OF THE DOCUMENT BEGINS.				
\begin{document}

\begin{center}\noindent{\bf Math 333:  Homework \#5}\\Mingchen Li \\ \end{center}
\thispagestyle{empty}



\hrulefill %This just makes a nice horizontal line. Useful if you like to separate your problems with lines!

In the first three problems, you will prove Cayley's Theorem, which states that ANY group can be viewed as a group of permutations. These three problems go together; all notation  carries over.

\begin{enumerate}



\item[{\bf Problem 1:}] Let $G$ be a group, and $g$ an element of $G$. Define a function $T_g:G\rightarrow G$ by $T_g(x) = gx$. Prove that $T_g$ is a permutation on the set of elements in $G$ (that is, prove that $T_g$ is a bijection).



\begin{enumerate}
    \item[Surjection:]
    Take arbitrary $y\in G$, we aim to find $x\in G$ such that $T_g(x)=y$. Since $g\in G$, $g^{-1}\G$ as well. 
    
    Since group operation is closed, $g^{-1}y\in G$ as well. Let $x=g^{-1}y$, given that group operation is associative, $T_g(x)=g(g^{-1}y)=(gg^{-1})y=y$. Thus we have find our x.
    \item[Injection:]
    Assume that there are $x_1\neq x_2\in G$ such that $T_g(x_1)=T_g(x_2)=y$. Then $gx_1=gx_2=y$. Apply $g^{-1}$ to both left side of the equation: 
    \[g^{-1}gx_1=x_1=x_2=g^{-1}gx_2\]
    
    This contradict our assumption and $T_g$ is hereby injection and surjective.
\end{enumerate}
    

\newpage
\item[{\bf Problem 2:}]  Now, let $\mathcal T_G = \{T_g\,|\, g\in G\}$. Prove that $\mathcal T_G$ forms a group under composition of functions.
\begin{enumerate}
    \item[Exist Identity: ] Consider $T_e \in \mathcal T_G $ where $T_e(x)=ex=x$. For arbitrary $g\in G, T_g\in \mathcal{T}_G$, 
    \[T_gT_e(x)=g(ex)=g(x)=e(g(x))=T_eT_g(x)\]
    Thus $T_e$ is the identity.
    \item[Exist inverse: ] Let $g\in G, T_g\in \mathcal{T}_G$ be arbitrary. Consider $T_{g^{-1}}$ to be the inverse of $T_g$ by using the associativiity of function composition:
    \[T_{g^{-1}}T_g(x)=g^{-1}g(x)= e(x)= T_e(x)=e(x)=gg^{-1}(x)=T_gT_{g^{-1}}(x)\]
    
    Thus we have found the inverse of the group.
    \item[Closed operation: ] Take arbitrary $g, t\in G$, $T_gT_t(x)= gt(x)$. Since G is closed under group operation, $gt\in G$. Thus $T_{gt}(x)=gt(x)=T_gT_t(x)$. Thus operation is closed.
    \item[Associativity: ] Take arbitrary $g, t, r\in G$, $T_gT_tT_r(x)=gtr(x)$. Since operation in G is associative, $T_g(T_tT_r)(x)=g(tr)(x)=(gt)r(x)=(T_gT_t)T_r(x)$. Thus operation in $\mathcal{T}_G$ is associative
\end{enumerate}
Thus $\mathcal{T}_G$ is a group under composition of functions.
\newpage
\item[{\bf Problem 3:}] Prove that $G\cong \mathcal T_G$. [This is {\bf Cayley's theorem:} you are proving that $G$ is isomorphic to a set of permutations....{\it of itself}.]
\begin{proof}
Consider the mapping: $\mathcal{F}: G \rightarrow \mathcal{T}_G$ where $\mathcal{F}(g\in G)=T_g$. We aim to show $\mathcal{F}$ is isomorphism: 
\newline
Bijection: Take arbitrary $T_g\in \mathcal{T}_G$ for some $g\in G$. By definition $\mathcal{F}(g)=T_g$. Thus $\mathcal{F}$ is Surjective.  Assume that there are $g_1\neq g_2\in G $ such that $\mathcal{F}(g_1)=\mathcal{F}(g_2)=T_g$. Then $T_{g_1}=T_{g_2}$ and $T_{g_1}(e)=g_1(e)=g_1=g_2=g_2(e)=T_{g_2}(e) \Rightarrow\Leftarrow$. Thus $\mathcal{F}$ is Injective.

\newline
Preserve operation: We aim to show:
\[\mathcal{F}(g_1g_2)=\mathcal{F}(g_1)\mathcal{F}(g_2)\]
We have shown before that $\mathcal{T}_G$ is a group. Thus $[\mathcal{F}(g_1g_2)](x)=T_{g_1g_2}(x)=T_{g_1}T_{g_2}(x)=\mathcal{F}(g_1)\mathcal{F}_{g_2}(x)$ for some $x\in G$. Thus $\mathcal{F}$ preserve operation and $G\cong \mathcal T_G$. 
\end{proof}




\newpage 
\item[{\bf Problem 4:}] Suppose $\varphi:G\rightarrow G'$ is an isomorphism of groups. 
\begin{itemize}
\item Prove that $G$ is Abelian if and only if $G'$ is Abelian.
\item Prove that $G$ is cyclic if and only if $G'$ is cyclic.
\end{itemize}

\begin{enumerate}
    \item  For arbitrary $\hat{g_1}, \hat{g_2}\in G'$, since $\varphi$ is an isomorphism, thereby surjective, there are $g_1, g_2 \in G$ such that $\varphi (g_1)=\hat{g_1}$ and $\varphi (g_2)=\hat{g_2}$. Since $\varphi$ is an isomorphism, thereby preserve operation:
    \[\varphi(g_1g_2)=\varphi(g_1)\varphi(g_2)=\hat{g_1}\hat{g_2}\]
    \[\varphi(g_2g_1)=\varphi(g_2)\varphi(g_1)=\hat{g_2}\hat{g_1}\]
    $\Rightarrow:$ Assume $G$ is Abelian, $g_1g_2=g_2g_1$, thus $\varphi(g_1g_2)=\varphi(g_2g_1)$. 
    \newline
    Thus we have $\hat{g_1}\hat{g_2}=\hat{g_2}\hat{g_1}$ and $G'$ is abelian.
    
    $\Leftarrow$ By property of isomorphism on groups, the inverse of the isomorphism is also isomorphism. Thus we can apply the same reasoning on $\varphi^{-1}$
    \item$\Rightarrow: $ For arbitrary $g'\in G'$, since $\varphi$ is surjective, there is some $g\in G$ such that $\varphi(g)=g'$. Assume G is cyclic, then $g=a^n$ where $G=\langle a \rangle$ and $\varphi(a)=a'$. Since $\varphi$ preserve operation:
        
        \[\varphi(g)=\varphi(a^n)=\varphi(a)^n=(a')^n=g'\]
        Thus $G'\subseteq \langle a' \rangle $. To show $G'\supseteq \langle a' \rangle $, let $m\in \N$ be arbitrary. Using the property of $\varphi$ shown above, we have $(a')^m=\varphi(a^n)\in G$. Thus $G'=\langle a' \rangle$
        
        $\Leftarrow$ By property of isomorphism on groups, the inverse of the isomorphism is also isomorphism. Thus we can apply the same reasoning on $\varphi^{-1}$
    
    
\end{enumerate}


\newpage
\item[{\bf Problem 5}] ( 6.44)Show that the mapping $a\mapsto \log_{10}(a)$ is an isomorphism from $\R_{>0}^*$, the group of positive real numbers under multiplication, to $\R$, the group of real numbers under addition. 
\begin{proof}
To show the mapping (denoted as f) is isomorphism, we need to show it's bijection and preserve operation.
\begin{enumerate}
    \item Bijection: Take arbitrary $x\in \R$, consider a=$10^x$, $a\in \R_{>0}^* $, we know $f(10^x)=log_{10}(10^x)=x$. Thus $f$ is surjective. Assume there are $a_1, a_2\in \R_{>0}^*$ such that $f(a_1)=f(a_2)=log_{10}(a)$. Since log is injective, $a_1=a_2$. Thus f is injective. 
    \item Preserve order: Take arbitrary $a, b\in \R_{>0}^*$, since log is distributive by product: $f(ab)=log_{10}(ab)=log_{10}(a)+log_{10}(b)=f(a)+f(b)$. Since the target group operation is addition, the operation is hereby preserved. 
\end{enumerate}
\end{proof}





\newpage
\item[{\bf Problem 6:}] Sort the following groups into equivalence classes under isomorphism, and prove that your sorting is correct.\footnote{To avoid having to do $\binom{10}{2}$ proofs, make use of previous problems as well as any results from class and from previous homework.}

\begin{enumerate}
\item $G=\left\{ \begin{bmatrix} 1 & a\\ 0 & 1\end{bmatrix} \, \bigg| \, a\in \Z\right\}$ under matrix multiplication.
\item $\R^* = \R\backslash\{0\}$ under multiplication
\item $\C^* = \C\backslash\{0\}$ under multiplication
\item $GL_2(\R)$ under matrix multiplication.
\item $\Z$ under addition
\item $6\Z$ under addition
\item $\Z/6\Z$ under addition mod 6
\item $\Z/2\Z \times \Z/3\Z$
\item The group of rotations of a regular hexagon 
\item $S_3$

\end{enumerate}
\begin{enumerate}
    \item[Class 1:] a, e, f
    \begin{proof}
    Define $\begin{bmatrix} 1 & a\\ 0 & 1\end{bmatrix}\mapsto a\in \Z$. The map is bijection because for all $a\in \Z$ we can construction a matrix $\begin{bmatrix} 1 & a\\ 0 & 1\end{bmatrix}$ that maps to it and this matrix is unique for obvious reason. It also preserve operation because $\begin{bmatrix} 1 & a\\ 0 & 1\end{bmatrix}\begin{bmatrix} 1 & b\\ 0 & 1\end{bmatrix}=\begin{bmatrix} 1 & a+b\\ 0 & 1\end{bmatrix} \mapsto a+b $ while $\begin{bmatrix} 1 & a\\ 0 & 1\end{bmatrix}\mapsto a$ and $\begin{bmatrix} 1 & a\\ 0 & 1\end{bmatrix}\mapsto b$. 
    
    Define $a\in \Z \mapsto 6a \in 6\Z$. This map is bijection because for each $6a\in 6\Z$, we have $a$ to map to it and a is unique since division by integer is injective. It also preserve operation because $a+b\mapsto 6a+6b$ while $a\mapsto 6a,  b\mapsto 6b$ for arbitrary $a,b \in \Z$
    
    $a\cong f$ because of the transitivity of isomorphism. Thus we have evaluated group 1.
    
    Group 1 is disjoint from other groups because they are the only groups that are countably infinite.
    
    \end{proof}
    
    \item[Class 2:] g, h, i
    \begin{proof}
    $g\cong i$: Consider a regular hexagon. Define the map $\varphi(r)=$ rotation by $60*a$ degree clockwise. This map is bijection: for each possible rotation on a hexagon, it must be a multiple of 60 degree. Thus we can find a unique one in $\Z/6\Z$ that maps to this rotation. This map preserve operation because $\varphi(a+_6b)=\text{rotate by (a+b)*60 degree}=\text{rotate by a*60 degree then b*60 degree}=\varphi(a)\varphi(b)$.
    
    $g\cong h$: define the map $\varphi(1)=(1,1)$. Since $\Z/6\Z, \Z/2\Z, \Z/3\Z$ are cyclic, we have mapped the generator to generator. To preserve operation we define the rest of mapping like this:
    
    \begin{enumerate}
        \item $\varphi(2)=(0,2)$
        \item $\varphi(3)=(1,0)$
        \item $\varphi(4)=(0,1)$
        \item $\varphi(5)=(1,2)$
        \item $\varphi(0)=(0,0)$
        
    \end{enumerate}
    Thus clearly this is a bijection, and $\varphi(a+b)=\varphi(1)^{a+b}=\varphi(1)^a\varphi{1}^b=\varphi(a)\varphi(b)$. Thus preserve operation.
    \end{proof}
    
    \item[Rest of the groups:] Group b, c  and d has higher cardinality than any of the class 1 and class 2 elements. Thus there will not be a bijection between them and class 1 or 2 element. Group b and c are not isomorphic because consider element $i\in \C^*, |i|=4$. However every element in $\R^*$ has infinite order. So no isomorphism can be established. In similar vein $GL_2(\R)$ have finite order element as well. $GL_2(\R)$ is not abelian while $C^*$ is. So no isomorphism can be established. 
    
    Group $S_3$ is not abelian as well while element in class 2 are, so no isomorphism can be established.
    
\end{enumerate}
\newpage
\item[{\bf Problem 7:}] Let $G$ be a group and let $Z(G)$ be the center of $G$ (i.e, $Z(G) = \{g\in G\,|\, gx=xg \text{ for all } x\in G\}$. Prove that $G/Z(G)$ is cyclic if and only if $G$ is Abelian.
\begin{proof}
$\Rightarrow$ : Assume that $G/Z(G)$ is cyclic, $G/Z(G)=\langle aZ(G) \rangle $ where $a\in G$. Then take arbitrary $x,y\in G$, we know $x\in xZ(G)=(aZ(G))^i=a^iZ(G)$. Thus $x=a^iz_x, z_x\in Z(G)$. In similar vein, we can denote $y=a^jz_y \in a^jZ(G)=yZ(G)$. Since $z_x, z_y$ commutes to all element in G:

\[xy= a^iz_x a^jz_y=a^ia^jz_x z_y=a^ia^jz_yz_x= a^ja^iz_yz_x=a^jz_ya^iz_x=yx\]
Hereby G is abelian.

$\Leftarrow$: Assume that G is Abelian, then for all $g\in G$, $xg=gx$ for all $x\in G$. Thus $G\subseteq Z(G)$ and $Z(G)\subseteq G$ by defintion. Thus $Z(G)=G$. Then we aim to show that $G/Z(G)=\{eZ(G)\}$ and hereby cyclic. For aribitrary $g\in G$, according to the property of cosets, $gZ(G)=Z(G)$ iff $g\in Z(G)$. Now we know that $Z(G)=G$, thus by definition $G/Z(G)=\{eZ(G)\}$.
\end{proof}
\newpage

\item[{\bf Problem 8:}]
Quotient groups are a powerful tool for algebraic proofs. In particular, since they allow us to ``reduce" a group down without destroying all of its structure, they provide us with a natural way to use induction on groups.

 In this problem, you will use quotient groups and induction to prove {\bf Cauchy's theorem:}  If $G$ is a finite Abelian group and $p$ is a prime dividing the order of $G$, then $G$ has an element of order $p$. 

The following steps will guide you through the proof. {\bf However, the final version should be written up as a single coherent proof, not as a series of answers to steps.}

\begin{itemize}
\item This proof is done by strong induction on $|G|$. So first, prove the base case ($|G|=2$), and state your inductive hypothesis.

\item Let $G$ be a group with $|G|>2$ and let $p$ be a prime dividing the order of $G$.  First, prove that $G$ must have an element of {\it some} prime order $q$ (it may be that $q\neq p$). You can do this directly (no induction yet).

\item If $q=p$, we are done, so assume $q\neq p$, and let $x$ be an element of $G$ order $q$. Show that: 
\begin{itemize}
\item $G'=G/\langle x\rangle$ is an Abelian group.
\item $p$ divides $|G'|$ and $|G'|<|G|$.
\end{itemize}

\item Apply the inductive hypothesis to obtain an element $y\langle x\rangle$ of $G'$ of order $p$. Show that either $y$ has order $p$, or $y^q$ has order $p$.


\end{itemize}
\begin{proof}
We shall prove this theorem using strict induction. First we aim to show for $|G|=2$, G is a finite abelian group, and p is a prime dividing the order of G, then G has an element of order p. In this case, the only prime possible is 2. For $G=\{e, a\}$, $|a|=2$ since if $aa= a$, then $a=e$ which contradict the assumption. Thus the statement is true for $|G|=2$.

Then we assume that the statement is true for all finite abelian group with order less than $|G|$, $|G|>2$. We aim to show that the statement is also true for abelian group $G$. 

Let $G$ be a group with $|G|>2$ and let $p$ be a prime dividing the order of $G$. Firstly the group $G$ must have some element of prime order $q$. For arbitrary $x\in G$, $|x|=qn$ where $q$ is prime and $n\in \N$. Then by the property of the group, $|x^n|=q$ and we have found the element with prime order q.

If $p=q$, we are done. Assume $p\neq q$. and denote $x$ to be the element of a group G that has order q. Define: $G'=G/\langle x\rangle$. For arbitrary $x,y \in G'$, $x=g_a\langle x\rangle, y=g_b\langle x\rangle$ for some $g_a, g_b\in G$. Since G is abelian, x, y are hereby normal:

\[xy= g_a\langle x\rangle g_b\langle x\rangle=g_a g_b\langle x\rangle= g_bg_a\langle x\rangle =g_b\langle x\rangle g_a\langle x\rangle=yx\]

Thus $G'$ is abelian, it can be applied to our induction assumption. By Lagrange's theorem:
\[|G/\langle x\rangle||\langle x\rangle|=|G|\]
Since $|G|=mp, |\langle x\rangle|=q$, $p$ divides $|G'|$ and by induction assumption, there is an element in $G'$, denoted as $y\langle x\rangle$, that has order of p. Thus:
\[(y\langle x\rangle)^p=y^p\langle x\rangle=\langle x\rangle\]
Thus $y^p\in\langle x\rangle$. Either $y^p=e$, or $y^p$ has order q and we can find $y^q$ to be the element in G that have order p.

\end{proof}

\end{enumerate}

\end{document}  

%THIS IS WHERE THE DOCUMENT ENDS. Anything written after this will not appear on the pdf.