\documentclass[11pt, oneside]{article}   	
\usepackage[margin=1in]{geometry}                		
\geometry{letterpaper}                   		
\usepackage{graphicx}						
\usepackage{amssymb}
\usepackage{amsmath}
\usepackage{amsthm}
\usepackage{enumerate}
\graphicspath{ {./hw4pic/} }


%Commands above this line set up the type of document, and ensure it has access to the LaTeX files needed to understand your commands.

%Here, I define some "shortcuts" for notation I commonly use. 
\newcommand{\N}{\mathbb N}
\newcommand{\Z}{\mathbb Z}
\newcommand{\Q}{\mathbb Q}
\newcommand{\C}{\mathbb C}
\newcommand{\R}{\mathbb R}
\newcommand{\F}{\mathbb F}

\newtheorem*{proposition}{Proposition}
\newtheorem*{theorem}{Theorem}

\title{Homework 1}
\author{The Author} 
%\date{}			




%THIS IS WHERE THE ACTUAL TEXT OF THE DOCUMENT BEGINS.				
\begin{document}

\begin{center}\noindent{\bf Math 381:  Homework \#5}\\Mingchen Li\\ \end{center}

\begin{enumerate}
\item[Problem 1]
\begin{itemize}
    
    \item[3.17] Using the appendix E, we have the following probabilities:
    \begin{enumerate}
        \item $P(X>3.5)=P(Z>\dfrac{3.5-(-2)}{\sqrt{7}})=P(Z>2.07)=1-\Phi(2.07)\approx 0.0192$
        \item $P(-2.1<X<-1.9)=P(\dfrac{-2.1+2}{\sqrt{7}}<Z<\dfrac{-1.9+2}{\sqrt{7}})=\Phi(-1.9)-\Phi(0.0377)=(1-\Phi(1.9))-(1-\Phi(2.1))= $
        \item $P(X<2)$
        \item $P(X<-10)$
        \item $P(X>4)$
    \end{enumerate}
       
    \item[3.18]
\end{itemize}
\end{enumerate}

\end{document}
