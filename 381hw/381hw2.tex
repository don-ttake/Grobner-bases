\documentclass[11pt, oneside]{article}   	
\usepackage[margin=1in]{geometry}                		
\geometry{letterpaper}                   		
\usepackage{graphicx}						
\usepackage{amssymb}
\usepackage{amsmath}
\usepackage{amsthm}
\usepackage{enumerate}

%Commands above this line set up the type of document, and ensure it has access to the LaTeX files needed to understand your commands.

%Here, I define some "shortcuts" for notation I commonly use. 
\newcommand{\N}{\mathbb N}
\newcommand{\Z}{\mathbb Z}
\newcommand{\Q}{\mathbb Q}
\newcommand{\C}{\mathbb C}
\newcommand{\R}{\mathbb R}
\newcommand{\F}{\mathbb F}

\newtheorem*{proposition}{Proposition}
\newtheorem*{theorem}{Theorem}

\title{Homework 1}
\author{The Author} 

\begin{document}
\begin{center}\noindent{\bf Math 381:  Homework \#2}\\Mingchen Li\\ \end{center}

\begin{enumerate}
    \item [{\bf Problem 1: ASV 1.15}] 
    \begin{enumerate}
        \item Using the notation given, P(we did not see all three colors) = $P(W\cup G \cup R)$. Using inclusion-exclusion we have following equation: 
        \[P(W\cup G \cup R)= P(W)+P(G)+P(R)-P(W\cap G)-P(R\cap G)-P(W\cap R)+P(W\cap G \cap R)\]
        Where we can calculate each component individually:
        
        $P(W)= P(G)= \dfrac{3^3}{4^3}= \dfrac{27}{64}$
        
        $P(R)=\dfrac{2^3}{4^3}=\dfrac{8}{64}$
        
        $P(W\cap G)=\dfrac{8}{64}$
        
        $P(W\cap G)=a\dfrac{8}{64}$
        
    \end{enumerate}
\end{enumerate}
\end{document}